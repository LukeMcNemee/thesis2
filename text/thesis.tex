%%%%%%%%%%%%%%%%%%%%%%%%%%%%%%%%%%%%%%%%%%%%%%%%%%%%%%%%%%%%%%%%%%%%
%% I, the copyright holder of this work, release this work into the
%% public domain. This applies worldwide. In some countries this may
%% not be legally possible; if so: I grant anyone the right to use
%% this work for any purpose, without any conditions, unless such
%% conditions are required by law.
%%%%%%%%%%%%%%%%%%%%%%%%%%%%%%%%%%%%%%%%%%%%%%%%%%%%%%%%%%%%%%%%%%%%

\documentclass[
  digital, %% This option enables the default options for the
           %% digital version of a document. Replace with `printed`
           %% to enable the default options for the printed version
           %% of a document.
  table,   %% Causes the coloring of tables. Replace with `notable`
           %% to restore plain tables.
  nolof,     %% Prints the List of Figures. Replace with `nolof` to
           %% hide the List of Figures.
  nolot,     %% Prints the List of Tables. Replace with `nolot` to
           %% hide the List of Tables.
  %% More options are listed in the user guide at
  %% <http://mirrors.ctan.org/macros/latex/contrib/fithesis/guide/mu/fi.pdf>.
]{fithesis3}
%% The following section sets up the locales used in the thesis.
\usepackage[resetfonts]{cmap} %% We need to load the T2A font encoding
%\usepackage[T1,T2A]{fontenc}  %% to use the Cyrillic fonts with Russian texts.
\usepackage[
  main=english, %% By using `czech` or `slovak` as the main locale
                %% instead of `english`, you can typeset the thesis
                %% in either Czech or Slovak, respectively.
  %german, russian, czech, slovak %% The additional keys allow
]{babel}        %% foreign texts to be typeset as follows:
%%
%%   \begin{otherlanguage}{german}  ... \end{otherlanguage}
%%   \begin{otherlanguage}{russian} ... \end{otherlanguage}
%%   \begin{otherlanguage}{czech}   ... \end{otherlanguage}
%%   \begin{otherlanguage}{slovak}  ... \end{otherlanguage}
%%
%% For non-Latin scripts, it may be necessary to load additional
%% fonts:
\usepackage{paratype}
\def\textrussian#1{{\usefont{T2A}{PTSerif-TLF}{m}{rm}#1}}
%%
%% The following section sets up the metadata of the thesis.
\thesissetup{
    date          = \the\year/\the\month/\the\day,
    university    = mu,
    faculty       = fi,
    type          = mgr,
    author        = Lukáš Němec,
    gender        = m,
    advisor       = {RNDr. Petr Švenda, Ph.D.},
    title         = {Edu-hoc: Experimental and educational platform for wireless ad-hoc networking},
    TeXtitle      = {Edu-hoc: Experimental and educational platform for wireless ad-hoc networking},
    keywords      = {keyword1, keyword2, ...},
    TeXkeywords   = {keyword1, keyword2, \ldots},
}
\thesislong{abstract}{
    TODO
}
\thesislong{thanks}{
    TODO
    %TODO podekovat za oblizovani
}
%% The following section sets up the bibliography.
\usepackage{csquotes}
\usepackage[              %% When typesetting the bibliography, the
  backend=biber,          %% `numeric` style will be used for the
  style=numeric,          %% entries and the `numeric-comp` style
  citestyle=numeric-comp, %% for the references to the entries. The
  sorting=none,           %% entries will be sorted in cite order.
  sortlocale=auto         %% For more unformation about the available
]{biblatex}               %% `style`s and `citestyles`, see:
%% <http://mirrors.ctan.org/macros/latex/contrib/biblatex/doc/biblatex.pdf>.
\addbibresource{bibliography.bib} %% The bibliograpic database within
                          %% the file `example.bib` will be used.
\usepackage{makeidx}      %% The `makeidx` package contains
\makeindex                %% helper commands for index typesetting.
%% These additional packages are used within the document:
\usepackage{paralist}
\usepackage{amsmath}
\usepackage{amsthm}
\usepackage{amsfonts}
\usepackage{url}
\usepackage{menukeys}
\begin{document}


%% We will define several mathematical sectioning commands.
%\newtheorem{theorem}{Theorem}[section] %% The numbering of theorems
                               %% will be reset after each section.
%\newtheorem{lemma}[theorem]{Lemma}     %% The numbering of lemmas
%\newtheorem{corr}[theorem]{Corrolary}  %% and corrolaries will
                                %% share the counter with theorems.
%\theoremstyle{definition}
%\newtheorem{definition}{Definition}
%\theoremstyle{remark}
%\newtheorem*{remark}{Remark}

\chapter{Introduction}
\chapter{ Problem analysis (testbed, not general WSN)}
  \section{Creating WSN network}
  \section{Possible challenges}
\chapter{TESTBED deployment}
  \section{Network design}
  \section{JeeTool (mass managment and communication)}
  \section{HW (Arduino, JeeNodes, RF12B radio ...)}\label{sec:hw}
\chapter{Research use}
  \section{Keys from radio signal}
    \subsection{4.1.1 Quantization principle (bits from signal strength)}
    \subsection{RSSI version}
    \subsection{CSI (channel state) version}
  \section{Cooperative jamming (can it improve our situation?)}
  \section{Performance Evaluation (results from experiments)}
    \subsection{Enthropy of data}
    \subsection{Speed (bits of key per time)}
    \subsection{Possible errors}
  \section{Discussion, is it achievable and under what conditions?}
\chapter{Education use}

  \section{motivation for educational WSN network}
  The current state of the art WSN devices usually uses specialized hardware
  and software in order to achieve the best performance available. This,
  unfortunately, is not the ideal prerequisite for an easy to learn matter.
  In fact, most of WSN devices have rather complicated setup and are
  quite challenging for novices.

  Because of such discouragement, it is difficult to teach how to
  work with WSN’s; few hours (at least) are usually required
  to explain the basics, which is reasonable for research project or
  something similar, but for class exercise, this would turn out to be
  not the most effective use of time, if it would be achievable at all. And
  we have not yet mentioned more advanced topics in this area, such as
  common encryption or message authentication techniques.

  Issue of this nature can be solved in various ways, %possible citations?
  in case of Edu-hoc we decided to sacrifice performance \ref{sec:hw}; which is not that much important for network with educational purpose. On the other hand,
  \section{Scenario approach (attack and repair) + iterative higher difficulty}

    \subsection{scenarios} - 5.2.5 individual scenarios
  \section{Evaluation principle}
  \section{Web interface and auto run}
  \section{PA197 use and results}
\chapter{Summary}




%%bibliography
%%==============================================================
{\csname captions\languagename\endcsname %% Temporarily override
%% the BibLaTeX localization with the original babel definitions.
\makeatletter %% Use the correct localization of the quotations.
  \thesis@selectLocale{\thesis@locale}\makeatother
\printbibliography[heading=bibintoc]} %% Print the bibliography.
%%==============================================================

\appendix %% Start the appendices.
\chapter{An appendix}

\end{document}
